\necislovana{Úvod} Časoprostorová data jsou pro mnoho lidí velice obecný pojem, pod kterým si často nedovedou nic konkrétního představit. Takto označovaná data jsou v dnešní době již běžně používána ve velké škále oborů od archeologie, geodézie, geologie, stavební inženýrství až po armádní účely, zemědělství, nebo územní plánování. Nejedná se tedy v žádném případě o specializovaný termín používaný pouze v jednom vědním oboru. 

Práci s časoprostorovými daty lze rozdělit na dvě skupiny. Na jedné jsou odborníci, kteří časoprostorová data sbírají, provádějí analýzy a vyhodnocují je pomoci specializovaných programů. Na druhé straně je poté široká veřejnost, která používá pouze finální produkt jejich práce. 

V dřívější době byla možnost publikace těchto dat jen velice omezena. To se rozšířením internetu a nástupem elektronicky publikovaných map pomoci publikačních platforem v poslední době rapidně změnilo. Právě Publikační platformy vyplňují mezeru mezi odborníky a širokou veřejností a dávají téměř každému možnost prezentace dat v ucelené formě. 

Jednou z takových publikačních platforem je také Gisquick. V tomto případě se navíc jedná o interaktivní platformu, kde může uživatel nejenom na data nahlížet, ale také v omezeném rozsahu měnit způsob jejich vizualizace. Ke stávajícím datům se rovněž dají přidat podkladové vrstvy, které uživateli zlepší celkový pohled na zobrazovanou situaci.

Spojení publikační platformy a časoprostorových dat je potom dalším krokem, který možnost prezentace dat ještě povýší. Pomocí vhodně vytvořených nástrojů pro filtrování dat na základě jejich časové osy se dají vytvářet simulace zobrazující průběh určitého vlivu v čase na určitém místě. Tímto způsobem se dá například jednoduchým způsobem simulovat vývoj zalesnění, migrace zvěře, nebo expanze měst. Pro uživatele lepšího časového vjemu lze také dosáhnout, vzhledem k charakteru dat, vytvořením animace, kdy se zobrazovaný stav mění s předem nastaveným časovým krokem. 
Výhodou při publikaci časoprostorových dat na platformě Gisquick je také jejich univerzálnost. Nezáleží tedy, zdali jsou data s odstupem milisekund či měsíců. Díky jednoduchému nastavení lze nástroj přizpůsobit téměř jakémukoli časovému intervalu. Velká benevolence je rovněž ve vstupním formátu časového řetězce.  


\bigskip
Na platformu Gisquick resp. její rozšíření o podporu pro časoprostorová data se zaměřuje obsah této práce. Hlavní téma je zaměřeno na implementaci rozhraní pro podporu pro zobrazování vektorových dat. V práci je podrobně popsán nástroj pro práci s časoprostorovými daty, jeho jednotlivé části a postup práce na jeho implementaci na platformu Gisquick, která je psána programovacím jazykem JavasScript s použitím frameworku Vue.js. 

S rozšířením platformy také úzce souvisí výzkum možných metod použitelných pro platformu Gisquick, který je popsán v první části práce. Je zde rovněž rozebrán způsob práce s časoprostorovými daty pro webové platformy obecně a stručný popis jejich základního fungování.

Poslední část je zaměřena na rozšíření zásuvného modulu pro QGIS. Do něj bylo vloženo nutné uživatelské rozhraní umožňující uživateli zvolit pro jednotlivé vrstvy jejich časový atribut tj. atribut obsahující časovou hodnotu. V této závěrečné části je popsán samotný plugin, jeho funkce v publikaci projektu a postup, kterým je jeho bylo uživatelské rozhraní doplněno. 
