\necislovana{Úvod} Časoprostorová data jsou pro mnoho lidí velice obecný pojem, pod kterým si často nedovedou nic konkrétního představit. Takto označovaná data jsou však v dnešní době běžně používána ve velké škále oborů od archeologie, geodézie, geologie, stavební inženýrství až po armádní účely, zemědělství, nebo územní plánování. Nejedná se tedy v žádném případě o specializovaný termín v jednom vědním oboru. 

Z tohoto ohledu lze práci s časoprostorovými daty rozdělit na dvě skupiny. Na jedné jsou odborníci, kteří časoprostorová data sbírají, provádějí analýzy a vyhodnocují je pomoci specializovaných programů. Na druhé straně je poté široká veřejnost, která používá pouze finální produkt jejich práce. 

V dřívější době byla však možnost publikace těchto dat jen velice omezena. To se rozšířením internetu a nástupem elektronicky publikovaných map pomoci publikačních platforem v poslední době rapidně změnilo. Právě Publikační platformy vyplňují mezeru mezi odborníky a širokou veřejností a dávají téměř každému možnost prezentace dat v ucelené formě. 

Právě jednou z takových publikačních platforem je také Gisquick. V tomto případě se navíc jedná o interaktivní platformu, kde může uživatel nejenom na data nahlížet, ale také v omezeném rozsahu měnit způsob jejich vizualizace. Ke stávajícím datům se rovněž dá přidat podkladový vrstva, což mnohdy uživateli usnadní orientaci.

Spojení publikační platformy a časoprostorových dat je potom dalším krokem, který celkovou prezentaci dat ještě umocní. Pomocí vhodně vytvořených nástrojů pro filtrování dat na základě jejich časové složky se dají vytvářet simulace zobrazující průběh určitého vlivu v čase na určitém místě. Tímto způsobem se dá například jednoduchým způsobem simulovat vývoj zalesnění, migrace zvěře, nebo vývoj měst. Pro lepší časový vjem je také možné, vzhledem na charakter dat, vytvářet animace, kdy se zobrazovaný stav mění s nastaveným krokem. 
Výhodou při publikaci časoprostorových dat na platformě Gisquick je také jejich univerzálnost. Je tedy jedno, jestli jsou data s odstupem milisekund, či měsíců, nebo v jakém jsou formátu.  


\bigskip
The main topic of this thesis is process isolation in PyWPS framework. A process is just some geospatial operation which 
has its defined inputs and outputs and which is deployed on a server. The server is able to execute multiple 
processes at the same time. This thesis deals with the isolation of individual processes, especially for security and 
performance reasons. With every process fully isolated, so they cannot interact with each other, the higher security 
level is ensured.

The thesis is composed of several parts. The first part describes the WPS standard, its operations 
\textit{GetCapabilities}, \textit{DescribeProcess} and \textit{Execute} and inputs and outputs structures. A quick
overview of some implementations of WPS standard follows and brings a basic information about them.

Nevertheless, this work is dedicated to PyWPS, an implementation in
Python. In the second part, its current state is described as well as
\textit{pywps-demo} - a side project providing demo server instance -
%% ML: and covers? opet divne formulovana veta
%% AL: prepsano. To cover - ve vyznamu pojednavat.
which the practical part is based on. Following research 
covers various projects and technologies which were considered as a
solution for process isolation. Eventually, the Docker technology is
chosen for the implementation part.  Docker has been selected as one
of the most used technology for containerization. It puts every
process into a separate container so the isolation is
ensured. Moreover, Docker provides a mechanism to pause, stop and
start a container so it looks like a possible solution for the future
WPS 2.0.0 standard implementation which requires this
functionality. Using Docker, it also opens new possibilities,
e.g. being able to deploy running job to cloud.

The third part describes the implementation. It explains the Execute
operation workflow, a process execution and how the Docker containers
are used for the PyWPS process isolation. New \textit{Container}
%% ML: were -> was?
%% AL: was
class, which was developed during the work on this thesis, is
introduced as well as its methods.
