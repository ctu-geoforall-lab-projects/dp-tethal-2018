\necislovana{Úvod}

Časoprostorová data jsou pro mnoho lidí velice obecný pojem, pod
kterým si často nedovedou nic konkrétního představit. Takto označovaná
data jsou v dnešní době již běžně používána ve velké řadě oborů od
archeologie, geodézie, geologie, stavebního inženýrství, územního
plánování až po armádní účely, zemědělství a meterologii. Nejedná se
tedy v žádném případě o specializovaný termín používaný pouze v jednom
vědním oboru.

%% ML: Na jedne strane?
%% DT: doplněno
Práci s časoprostorovými daty lze rozdělit na dvě skupiny. Na jedné straně
jsou odborníci, kteří časoprostorová data sbírají, provádějí analýzy,
vyhodnocují je pomocí specializovaných programů a vytvářejí nad nimi
tématické mapy. Na druhé straně je poté široká veřejnost, která
používá pouze finální produkt jejich práce.

V dřívější době byla možnost publikace těchto dat jen velice
omezena. To se v dnešní době s rozšířením internetu a nástupem
elektronicky publikovaných map pomoci publikačních platforem rapidně
%% ML: male p anebo do uvozovek
%% DT: opraveno
změnilo. Právě publikační platformy vyplňují mezeru mezi odborníky a
širokou veřejností tím, že dávají téměř každému možnost prezentace dat v
ucelené formě.


Spojení publikační platformy a časoprostorových dat je potom dalším
krokem, který možnost prezentace dat ještě umocní. Pomocí vhodně
vytvořených nástrojů pro filtrování dat na základě jejich časové osy
se dají vytvářet simulace zobrazující průběh určitého vlivu v čase na
určitém místě. Tímto způsobem se dá například jednoduchým způsobem
simulovat vývoj požárů, migrace zvěře, nebo expanze měst. Pro
uživatele lze rovněž dosáhnout lepšího časového vjemu, vytvořením
názorné animace, kdy se zobrazovaný stav mění s předem nastaveným
časovým krokem.

Jednou z takových publikačních platforem je právě Gisquick. V tomto
případě se navíc jedná o interaktivní platformu, kde může uživatel
nejenom na data nahlížet, ale také měnit způsob jejich vizualizace. Ke
stávajícím datům se rovněž dají přidat podkladové vrstvy, které
uživateli zlepší celkový pohled na zobrazovanou situaci. Výhodou při
publikaci časoprostorových dat na platformě Gisquick je také jejich
univerzálnost. Nezáleží tedy, zda-li jsou data pořízena s odstupem minut či
roků. Díky jednoduchému nastavení lze nástroj přizpůsobit téměř
jakémukoli časovému intervalu. Velká benevolence je rovněž ve vstupním
formátu časového řetězce.

\bigskip
Na platformu Gisquick resp. její rozšíření o podporu pro
časoprostorová data se zaměřuje obsah této práce. Hlavní téma je
%% ML: Nekde bys mel explicitne napsat, ze rastrova data nejsou v
%% ramci prace resena
%% DT: doplněna veta
zaměřeno na implementaci rozhraní pro podporu zobrazování vektorových
dat obsahujících časovou složku. Rastrová data v podpoře zobrazování časoprostorových dat zahrnuta nejsou. V práci je podrobně popsán nástroj
pro práci s časoprostorovými daty, jeho jednotlivé části a samotná
implementace na platformě Gisquick.

Součástí implementace je také část zaměřená na rozšíření zásuvného
%% ML: vlozeno -> pridano?
%% DT: pridano
modulu pro program QGIS. Do něj bylo přidáno nezbytné uživatelské
rozhraní umožňující uživateli zvolit pro jednotlivé vrstvy jejich
časový atribut tj. atribut obsahující časovou hodnotu. V částí práce věnované zásuvnému modulu je popsána jeho funkce v publikaci
projektu a postup, kterým bylo jeho uživatelské rozhraní doplněno.

%% ML: dvakrat ve vete ``platforma''
%% DT: opraveno
S rozšířením o novou funkcionalitu také úzce souvisí výzkum možných metod práce s
časoprostorovými daty použitelných pro platformu Gisquick. Rešerše již
aplikovaných metod na webových mapových serverech a aplikacích je
obsažena v první části práce.