%\necislovana{Seznam příloh}

\newpage
\part{Přílohy}
\label{prilohy}
\appendix

\newpage
\section{Struktura elektronických příloh}
\label{struktura_příloh}

\setlength{\unitlength}{.5mm}
\begin{picture}(250, 220)
\put(  0, 212){\textbf{.}}
\put(  1, 200){\line(0, 1){5}}
\put(  1, 190){\line(0, 1){10}}
\put(  1, 190){\line(1, 0){10} {\textbf{ src}}}
\put( 16, 180){\line(0, 1){8}}
\put( 16, 180){\line(1, 0){10} {\textbf{ diff}}}
\put( 31, 170){\line(0, 1){8}}
\put( 31, 170){\line(1, 0){10} { klient.diff}}
\put(150, 170){Gisquick klient soubor diff}
\put( 31, 160){\line(0, 1){10}}
\put( 31, 160){\line(1, 0){10} { plugin.diff}}
\put(150, 160){Gisquick plugin soubor diff}
\put(  1, 140){\line(0, 1){50}}
\put(  1, 140){\line(1, 0){5} {\textbf{ text}}}
\put( 16, 130){\line(0, 1){8}}
\put( 16, 130){\line(1, 0){10} {\textbf{ LaTeX}}}
\put(150, 130){zdrojový kód programu LaTeX}
\put( 16, 120){\line(0, 1){10}}
\put( 16, 120){\line(1, 0){10} { david-tethal-dp-2018.pdf}}
\put(150, 120){text diplmové práce}
\put(  1, 100){\line(0, 1){50}} 
\put(  1, 100){\line(1, 0){10} {\textbf{ zadani}}} 
\put( 16, 90){\line(0, 1){8}}
\put( 16, 90){\line(1, 0){10} { zadanidp.pdf}} 
\put(150, 90){zadání diplomové práce}
\put(  1, 70){\line(0, 1){50}} 
\put(  1, 70){\line(1, 0){10} {\textbf{ vzorový-projekt}}} 
\put(150, 70){vzorový QGIS projekt}

\end{picture}

\newpage
\section{Uživatelský manuál}
\label{uzivatelsky_manual}

Uživatelský manuál se skládá ze dvou částí. První obsahuje postup pro instalaci a spuštění platformy Gisquick na lokálním zařízení. Druhá část obsahuje uživatelské rozhraní zásuvného modulu pro platformu QGIS a dále Gisquick webového klienta. Uživatelské rozhraní je popsáno pouze pro části, které obsahují rozhraní pro práci s časoprostorovými daty.
Vzhledem k mezinárodnímu použití platformy Gisquick je grafické uživatelské rozhraní psáno v anglickém jazyce.

\subsection{Instalace}
\label{manual_instalace}

\newpage
\subsection{Grafické uživatelské rozhraní}
\label{manual_gui}

Time support for Gisquick platform allows users to easily filter 
map content based on its spatio-temporal value. Any vector layer that 
contain attribute with time value may be used. 

This section contains user manual describing process of time layers 
publication in Gisquick plugin for QGIS together with functions of time
filtering tool in Gisquick klient.  

\bigskip
\noindent \textbf{Publication process}

There is small settings in order to keep 
publication process simple and easy to handle for any user. Time layer 
may be set up in the first page of the publication wizard in the tab \textit{Layers}.
Dropdown menu \textbf{Time Attribute} defines which layer attribute contains
time values. In case this column is left blank layer won't be listed in 
the time filter tool. Second option is \textbf{Date Mask}. Time values may 
have different formats in different countries. No matter how original 
time format looks like, date mask offers a possibility to customize 
displayed date format in Gisquick client.

\begin{figure}[h!]
	\centering
	\includegraphics[width=0.75\textwidth]{../img/plugin-layers.png}
	\caption{Time layers options in publication wizard}
	\label{fig:publication-wizard-layers}
\end{figure}

Every user should know how the data looks like. 
\textbf{Date Mask} changes user interface of the time filtering tool. 
E.g. if 'HH:mm' mask is selected date picker will not be 
displayed in the client side, only time picker. 

Its recommended to use time date in format starting with 
year e.g. 'YYYY-MM-DD'. Otherwise, new attribute containing time 
values in UNIX time format has to be created during publication 
process.

The \textbf{configuration summary} wizard page displays all the parameters 
that were computed for each time layer. Note that if the 
\textbf{Time Attribute} field was left blank, parameters will not have any 
value.

\begin{figure}[h!]
	\centering
	\includegraphics[width=0.7\textwidth]{../img/project-publishing-time-summary.png}
	\caption{Time layers publication summary}
	\label{fig:publication-wizard-summary}
\end{figure}

\bigskip
\noindent \textbf{Time filtering tool}

Time filtering tool in Gisquick client is available in the tool menu 
in upper left corner. 

\begin{figure}[h!]
	\centering
	\includegraphics[width=0.4\textwidth]{../img/burger-menu.png}
	\caption{Tool menu}
	\label{fig:burger-menu}
\end{figure}

\bigskip
When the tool is activated dropdown menu with time layers appears on 
the left side of map canvas. For needs of filtration the time layer 
has to be specified. Despite selecting one time layer there is
also possibility of selecting \textit{All visible layers}.

\begin{figure}[h!]
	\centering
	\includegraphics[width=0.4\textwidth]{../img/time-layers-dropdown.png}
	\caption{Dropdown menu containing time layers}
	\label{fig:time-layers-drpdown}
\end{figure}

\bigskip
It might happen that two time layers with different time extend are 
visible in the same time. E.g. one layer displaying map features 
over one hour and second over one year. Filtering this two layers 
together using one time slider would ignore layer with shorter time 
extend. That is the reason why only time layers with same time 
attribute may be displayed in the same time. In case `All visible 
layers` contains different time attribute. User have to specify one 
in displayed dropdown menu.

\begin{figure}[h!]
	\centering
	\includegraphics[width=0.4\textwidth]{../img/time-attribute-dropdown.png}
	\caption{Dropdown menu containing time attributes}
	\label{fig:time-attribute-dropdown}
\end{figure}

\newpage
Once filtering layer is specified various settings is displayed. 

\begin{figure}[h!]
	\centering
	\includegraphics[width=0.4\textwidth]{../img/time-filtering-tool.png}
	\caption{Initialized time filtering tool}
	\label{fig:time-filtering-tool}
\end{figure}

Time filtering tool is composed of following parts:

\begin{itemize}
	\item\textit{Dropdown menu with time layers}
	\item\textit{Time attribute label}
	\item\textit{Double range slider}
	\item\textit{Lower and upper time value}
	\item\textit{Animation button with cumulative switch}
\end{itemize}

Function of \textbf{Dropdown menu with time layers} was mentioned before.
If selected time layer was already filtered, time filtering 
tool is initialized using previously used values.

\textbf{Time attribute label} displays name of attribute that was selected
in the process of project publishing as `Time Attribute`. This 
comes handy especially when `All visible layers` are selected.

\textbf{Double range slider} allows user to make fast data filtration.
Time interval is set using two sliders. Map content is refreshed 
each time the slider is changed. Step of double range slider is set 
as one hundredth of time interval size.

\begin{figure}[h!]
	\centering
	\includegraphics[width=0.4\textwidth]{../img/time-slider.png}
	\caption{Double range slider}
	\label{fig:time-slider}
\end{figure}

Time interval may be specified with better precision than time 
slider using \textbf{Lower and upper time value} labels. Precision depends
on selected output time format in project publication. If format 
contains time and date, then labels allow user to set time using time
pickers and date in calendar. In case that time format contains date 
than is not more precise than one day, only calendar is displayed. 
Time and date 
selection in displayed date time picker has to be confirmed by 
\textit{OK} button. Map canvas is updated after this confirmation.

\begin{figure}[h!]
	\centering
	\includegraphics[width=0.3\textwidth]{../img/date-time-picker.png}
	\caption{Date time picker}
	\label{fig:date-time-picker}
\end{figure}

Simple animation can be made using \textbf{Animation button}. There are two 
options. Classic and cumulative animation. Classical one increases both 
upper and lower values every second by slider step. 
Animation stops when upper value reach slider maximum. If cumulative 
mode is turned on only upper value is being increased. When it reaches 
slider maximum than lower value increases in the same pattern.
Animation ends when difference between upper and lower value is just one 
step. Second  way how to stop animation may be click on \textit{STOP} button. 
Stop button appears only when animation is on. Map canvas is updated 
each time when slider value is changed.

\begin{figure}[h!]
	\centering
	\includegraphics[width=0.4\textwidth]{../img/time-animation-stop.png}
	\caption{Time animation stop button}
	\label{fig:time-animation-stop}
\end{figure}

\newpage
\section{Seznam obrázků a tabulek}
\listoffigures
\listoftables


