\newpage
\necislovana{Závěr}

Hlavní cíle této práce byly dva. První spočíval v rešerši možných
metod pro řešení podpory časoprostorových dat pro webové mapové
publikační platformy obecně. Druhý cíl práce spočíval v implementování
%% ML: na -> pro?
%% DT: v 
nástroje pro podporu zobrazování vektorových dat v publikační
platformě Gisquick. Jednalo se především o úpravy na straně Gisquick
webového klienta a dále zásuvného modulu pro QGIS.

Zásuvný modul byl doplněn o jednoduché uživatelské rozhraní umožňující
snadné nastavení jednotlivých vrstev obsahujících časoprostorová
data. Jedná se především o definování atributu obsahujícího časové
hodnoty v nejrůznějších časových formátech. Na základě tohoto atributu
jsou posléze dopočítány další parametry nutné pro implementovaný
nástroj v rozhraní webového klienta.

Nejvíce změn především z uživatelského hlediska bylo provedeno na již
zmiňovaném webovém klientu. Zde byl přidán zcela nový nástroj, který
na základě časové osy nabízí uživateli kontrolu nad výběrem
zobrazovaných mapových prvků. Nástroj obsahuje různé způsoby určení
časového intervalu pro selekci mapových prvků jedné, nebo více
vrstev. Nástroj dále umožňuje vytvářet jednoduché animace.

Způsob jakým byl jak samotný nástroj, tak změny zásuvného modulu
implementovány úzce odráží poznatky získané během rešerše. Jedná se o
funkční řešení převzaté z časové podpory pro MapServer a
GeoServer. Stejnou mírou také o aplikované řešení z hlediska
uživatelského rozhraní použité v produktech firmy Esri. Konkrétně
ArcGIS Server a ArcGIS Online.

I přes vytvoření plně funkčního nástroje, který nabízí příjemné
uživatelské rozhraní a širokou škálu možností, je zapotřebí budoucího
vývoje nástroje. Nejdůležitější pro časový nástroj především z
hlediska jeho komplexnosti by byla implementace pro zobrazování
časoprostorových rastrových vrstev. Dále se jedná na straně zásuvného modulu
pro QGIS o zjednodušení validace dat a následného výpočtu,
které by bylo přínosné především pro projekty s větším objemem dat. V
této práci je aplikovaná validace dosti benevolentní, kdy na úkor časově
náročnějšího výpočtu dovoluje uživatelům použití dat s nekonzistentním
časovým formátem. Na straně webového klienta jde především o grafickou
stránku nástroje, která vzhledem k jeho možným budoucím změnám nebyla
brána jako důležitá.

\newpage
Pro potřeby vývoje rozšíření platformy Gisquick byly vytvořeny dva repozitáře. 
Repozitář \textit{dp-tethal-2018-gisquick}
obsahující mimo jiné webového klienta ve větvi \textit{vue-client}.
Dále repozitář \textit{dp-tethal-2018-gisquick-qgis-plugin} obsahující zásuvný 
modul pro program QGIS ve větvi \textit{master}
Oba vycházejí z původního repozitáře platformy Gisquick. Konkrétně 
repozitáře \textit{gisquick} resp. \textit{gisquick-qgis-plugin}
\footnote{původní repozitáře jsou dostupné z \url{https://github.com/gislab-npo/gisquick} a \newline \url{https://github.com/gislab-npo/gisquick-qgis-plugin}}
.
\newline
Rozdíly mezi původními a vývojovými repozitáři jsou uvedeny v 
souborech klient.diff resp. plugin.diff v elektronické příloze. 

%% ML: uved odkaz na git repozitar prace s odkazem na diff, v teto
%% souvislosti uved take repositare Gisquicku proti kterym je diff
%% relevantni
%% DT: doplněno